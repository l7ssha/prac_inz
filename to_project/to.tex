\documentclass[12pt,a4paper]{article}

\usepackage{minted}
\usepackage[T1]{fontenc}

\renewcommand*\contentsname{Spis treści}

\title{Stacja pogodowa oparta na mikrokontrolerze ESP32 z interfejsem użytkownika oraz API}
\author{Szymon Uglis}

\begin{document}

\begin{titlepage}
    \maketitle
\end{titlepage}

\tableofcontents{}
\pagebreak

\section{Wstęp}

// TODO:

\section{Cel projektu}

Celem projektu jest zbadanie przydatności modułu ESP32 oraz kompatybilnych komponentów do celów kolekcji danych pogodowych. 
Rezultatem będzie utworzenie urządzenia opartego na systemie wbudwanym pozwalającym na wielofunkcyjny pomiar parametrów pogodowych oraz udostępnianie ich za pomocą witryny www.

Urządzenie, będzie udostępniało również interfejs programistyczny API REST, które będzie umożliwiało integracje oraz dalszy rozwój projektu 
(np. integracja z systemem home assitant czy innymi urządzeniami Internetu rzeczy (IoT))

\section{Opis rozwiązania tworzonego w ramach projektu}

Narzędzie będzie składać sie z mikrokontrolera ESP-WROOM-32 oraz kilku czujników pozwalająych kolekcje danych pogodowych (takich jak temperatura, ciśnienie powietrza i inne)

\paragraph{Funkcje urządzenia:}
\begin{itemize}
    \item Pomiar oraz kalkulacja danych pogodowych na podstawie danych wejściowych z czujników
    \item Udostępnienie i agregacja danych w postaci strony www
    \item Udostępnienie interfejsu programistycznego REST 
\end{itemize}

\subsection{Wykorzystane technologie}
Do implemntacji systemu wbudowanego wykorzystane zostaną:
\begin{itemize}
    \item Język C do utworzenia oprogramowania mikrokontrolera
    \item Środowisko programistyczne Arduino Studio
    \item Język HTML oraz CSS do utworzenia interfejsu www
\end{itemize}

\subsection{Zastosowany mikrokontroler oraz czujniki}

\begin{itemize}
    \item Mikrokontroler ESP-WROOM-32
    \item Czujnik natężenia światła - TSL25911
    \item Czujnik temperatury i wilgotności - DHT22
    \item Czujnik ciśnienia oraz temperatury - DPS310
\end{itemize}

\subsection{Opis 7W}

// TODO: 

\subsection{Wymagania niefunkcjonalne}

// TODO: 

\subsection{Przypadki użycia}

\begin{itemize}
    \item Monitoring warunków pogodowych bez dodatkowych integracji
    \begin{itemize}
        \item Uruchomienie urządzenia w danym miejscu
        \item Monitoring zmian parametrów pogodowych za pomocą wbudowanego interfejsu www
    \end{itemize}
    \item Wykorzystanie interfejsu programistycznego REST API
    \begin{itemize}
        \item Uruchomienie urządzenia w danym miejscu
        \item Połączenie oraz pobranie danych z urządzenia za pomocą interfejsu REST w dowolnym języku programowania i technologii
    \end{itemize}
\end{itemize}


\end{document}
