\documentclass[12pt,a4paper]{article}

\usepackage{minted}
\usepackage[T1]{fontenc}

\renewcommand*\contentsname{Spis treści}

\title{Stacja pogodowa oparta na mikrokontrolerze ESP32 z interfejsem użytkownika oraz API}
\author{Szymon Uglis}

\begin{document}

\begin{titlepage}
    \maketitle
\end{titlepage}

\tableofcontents{}
\pagebreak

\section{Wstęp}

// TODO:

\section{Cel projektu}

Celem projektu jest zbadanie przydatności modułu ESP32 oraz kompatybilnych komponentów do celów kolekcji danych pogodowych. 
Rezultatem będzie utworzenie urządzenia opartego na systemie wbudwanym pozwalającym na wielofunkcyjny pomiar parametrów pogodowych oraz udostępnianie ich za pomocą witryny www.

Urządzenie, będzie udostępniało również interfejs programistyczny API REST, które będzie umożliwiało integracje oraz dalszy rozwój projektu 
(np. integracja z systemem home assitant czy innymi urządzeniami Internetu rzeczy (IoT))

\section{Opis rozwiązania tworzonego w ramach projektu}

Narzędzie będzie składać sie z mikrokontrolera ESP-WROOM-32 oraz kilku czujników pozwalająych kolekcje danych pogodowych (takich jak temperatura, ciśnienie powietrza i inne)

\paragraph{Funkcje urządzenia:}
\begin{itemize}
    \item Pomiar oraz kalkulacja danych pogodowych na podstawie danych wejściowych z czujników
    \item Udostępnienie i agregacja danych w postaci strony www
    \item Udostępnienie interfejsu programistycznego REST 
\end{itemize}

\subsection{Wykorzystane technologie}
Do implemntacji systemu wbudowanego wykorzystane zostaną:
\begin{itemize}
    \item Język C do utworzenia oprogramowania mikrokontrolera
    \item Środowisko programistyczne Arduino Studio
    \item Język HTML oraz CSS do utworzenia interfejsu www
\end{itemize}

\subsection{Zastosowany mikrokontroler oraz czujniki} \label{list_of_components}

\begin{itemize}
    \item Mikrokontroler ESP-WROOM-32
    \item Czujnik natężenia światła - TSL25911
    \item Czujnik temperatury i wilgotności - DHT22
    \item Czujnik ciśnienia oraz temperatury - DPS310
\end{itemize}

\subsection{Opis 7W}

\subsubsection{Why}
Urządzenie oraz aplikacja powstaje w celu zbioru oraz prezentacji danych pogodowych w formacie przyjaznym dla uzytkownika oraz w formacie, który
umożliwi integrację z innymi systemami.

\subsubsection{What}
Należy przygotować urządzenie umożliwiające kolekcję danych oraz ich udostępnianie oraz należy przygotować oprogramowanie, które będzie 
sterowało pracą urządzenia.

\subsubsection{When}

Wytwarzanie prototypu produktu zajmie 120 dni. Od 28.10.2023 do 28.02.2024.

\subsubsection{Who}

\begin{itemize}
    \item Klient
    \item Programista (1)
\end{itemize}

\subsubsection{Where}
Wrocławska Akademia Biznesu w Naukach Stosowanych, ul. Aleksandra Ostrowskiego 22, 53-238 Wrocław

\subsubsection{How}

Zostanie skonstruowane urządzenie za pomocą komponentów wymienionych w punkcie \ref{list_of_components} oraz zostanie napisane 
oprogramowanie sterujące w języku C. Produkt będzie wytwarzany w metodologii waterfall.

\subsubsection{How}

Wytwarzanie prototypu produktu zajmie 120 dni.

\subsection{Wymagania niefunkcyjne}

\begin{enumerate}
    \item Prostota instalacji urządzenia
    \item Możliwość zasilenia za pomocą kabla z końcówką USB-C
    \item Łatwość dostępu do danych dla końcowego użytkownika
    \item Dokumentacja do połączenia z API programistycznym
\end{enumerate}

\subsection{Wymagania funkcyjne}

\begin{enumerate}
    \item Podłączenie urządzenia oraz pierwsza konfiguracja\\
    Opis: Podłączenie urządzenia do źródła zasilania, wyszukanie punktu wifi urządzenia, wpisanie dostępów do wifi docelowego
    \item Odwiedzenie strony www zawierającej dane zebrane przez urządzenie\\
    Opis: Odwiedzenie strony www stworzonej przez urządzenie, odczytanie danych zebrnych przez urządzenie w formacie łatwo dostępnym dla
    użytkownika końcowego
\end{enumerate}

\subsection{Przypadki użycia}

\begin{itemize}
    \item Monitoring warunków pogodowych bez dodatkowych integracji
    \begin{itemize}
        \item Uruchomienie urządzenia w danym miejscu
        \item Monitoring zmian parametrów pogodowych za pomocą wbudowanego interfejsu www
    \end{itemize}
    \item Wykorzystanie interfejsu programistycznego REST API
    \begin{itemize}
        \item Uruchomienie urządzenia w danym miejscu
        \item Połączenie oraz pobranie danych z urządzenia za pomocą interfejsu REST w dowolnym języku programowania i technologii
    \end{itemize}
\end{itemize}

\section{Opis ryzyka}

\subsection{Potencjalne zagrożenia}

\begin{enumerate}
    \item Niedotrzymanie terminu\\
    Prawodopodobieństwo: małe\\
    Istnotność: katastrofalne\\
    Przyczyny:
    \begin{itemize}
        \item Słaba koordynacja prac
        \item Nieprzewidziany problem techniczny
    \end{itemize}
    Skutki:
    \begin{itemize}
        \item Zapłata kar umownych
        \item Strata klienta
    \end{itemize}
    \item Wyniki testów poniżej dopuszczalnej normy jakości
\end{enumerate}


\subsection{Plany awaryjne}

\begin{enumerate}
    \item Niedotrzymanie terminu
        \begin{enumerate}
            \item Zwiększenie liczby godzin pracy
            \item Zatrudnienie dodatkowych osób
        \end{enumerate}
    \item Wyniki testów poniżej dopuszczalnej normy jakości
        \begin{enumerate}
            \item Dokładna analiza modułów sprawiających najwięcej błędów
        \end{enumerate}
\end{enumerate}

\end{document}
