\documentclass[12pt,a4paper]{article}

\usepackage{minted}
\usepackage[T1]{fontenc}

\title{Porówanie implementacji aplikacji Android we toolkicie Flutter oraz natywym toolkicie dla Androida}

\begin{document}

\section{Wstęp}

Praca ma na celu porówanie implementacji identycznej aplikacji android we frameworku Flutter 
oraz natywnego toolingu do tworzenia aplikacji na platformę android. 

Porównanie to będzie uwzględniało różnice pomiędzy dwa toolkitami oraz praktyczne róznice w implementacji apliacji, takie jak: 
 * styl kodu
 * ilość kodu wymaganego do osiągnieca celu
 * rodzaj prac dodatkowych wymaganych do deployu aplikacji
 * łatwość dalszego rozwoju apliacji

\subsection{Opis aplikacji tworzonej w ramach projektu}

Aplikacja będzie pomocą dla ludzi interesujacych się fotografią analogową. Aplikacja będzie zawierała konfigurowalny czasomierz do odliczania czasu
podczas wywoływania filmu lub papieru fotograficznego w warunkach domowych.

\begin{enumerate}
    \item Pierwszy ekran - Konfiguracja czasomierzy
    \begin{itemize}
        \item Ilość czasomierz oraz ile czasu mają odmierzać
        \item Konfiguracja przerw między czasomierzami
        \item Przycisk do startu pracy czasomierzy
    \end{itemize}
    \item Drugi ekran - Praca czasomierzy
    \begin{itemize}
        \item Czasomierze pracują jeden po drugim, każdy odliczając tyle czasu ile zostało zkonfigurowane we wcześniejszym ekranie
        \item Pomiędzy każdym z czasomierzy, będzie odliczany cooldown zkonfigurowany we wcześniejszym ekranie
        \item Po zakończonej pracy wyświetli sie informacja o zakończeniu pracy, oraz aplikacja powróci do pierwszego ekranu
        \item Latający przycisk do anulowania wykonywania czasomierzy
    \end{itemize}
\end{enumerate}

\subsection{Wykorzystane technologie}
Do implementacji aplikacji w toolkicie Flutter zostajną wykorzystane:
\begin{itemize}
    \item Flutter
    \item Dart
\end{itemize}

Do implementacji aplikacji w toolkicie natywnym dla Androida zostajną wykorzystane:
\begin{itemize}
    \item Kotlin
    \item JDK
\end{itemize}

Do stworzenia dwóch aplikacji zostanie użyte środowisko Adroid Studio.

\end{document}
