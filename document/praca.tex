\documentclass[12pt,a4paper]{article}

\usepackage{minted}
\usepackage[T1]{fontenc}

\title{Porówanie implementacji aplikacji z narzędziami dla fotografa analogowego we frameworku Flutter do implementacji w Android NDK w języku Kotlin}

\begin{document}

\maketitle

\section{Wstęp}

Praca ma na celu porówanie implementacji identycznej aplikacji android we frameworku Flutter 
oraz implemntacji w języku Kotlin używając Android Native Development Kit.

Porównanie to będzie uwzględniało różnice pomiędzy dwa toolkitami oraz praktyczne róznice w implementacji apliacji, takie jak:
\begin{itemize}
 \item styl kodu
 \item ilość kodu wymaganego do osiągnieca celu
 \item rodzaj prac dodatkowych wymaganych do deployu aplikacji
 \item łatwość dalszego rozwoju apliacji
\end{itemize}

\subsection{Opis aplikacji tworzonej w ramach projektu}

Aplikacja będzie zbiorem narzędzi oraz pomocy dla ludzi interesujacych się fotografią analogową. Aplikacja będzie zawierała między innymi konfigurowalny czasomierz do odliczania czasu
podczas wywoływania filmu lub papieru fotograficznego w warunkach domowych, organizer do posiadanych oddczyników oraz materiałów forograficznych.

\begin{enumerate}
    \item Ekran powitalny z listą dostępnych narzędzi
    \begin{itemize}
        \item Czasomierz asystujacy w wywoływaniu filmów i papierów fotograficznych
        \item Organizer posiadanych materiałów i oddczynników forograficznych
    \end{itemize}
    \item Czasomierz
    \begin{itemize}
        \item Konfiguracja czasomierzy
        \begin{itemize}
            \item Ilość czasomierz oraz ile czasu mają odmierzać
            \item Konfiguracja przerw między czasomierzami
            \item Przycisk do startu pracy czasomierzy
        \end{itemize}
        \item Praca czasomierzy
        \begin{itemize}
            \item Czasomierze pracują jeden po drugim, każdy odliczając tyle czasu ile zostało zkonfigurowane we wcześniejszym ekranie
            \item Pomiędzy każdym z czasomierzy, będzie odliczany cooldown zkonfigurowany we wcześniejszym ekranie
            \item Po zakończonej pracy wyświetli sie informacja o zakończeniu pracy, oraz aplikacja powróci do pierwszego ekranu
            \item Latający przycisk do anulowania wykonywania czasomierzy
        \end{itemize}
    \end{itemize}
    \item Organizer materiałów oraz oddczynników fotograficznych
    \begin{itemize}
        \item Lista aktualnie zapisanych materiałów oraz oddczynników forograficznych
        \item Ekran dodawania nowego wpisu
        \item Ekran szczegółów danego wpisu 
    \end{itemize}
    \item 
\end{enumerate}

\subsection{Wykorzystane technologie}
Do implementacji aplikacji w toolkicie Flutter zostajną wykorzystane:
\begin{itemize}
    \item Flutter
    \item Dart
\end{itemize}

Do implementacji aplikacji w Android NDK wykorzystane:
\begin{itemize}
    \item Kotlin
    \item NDK
    \item JDK
\end{itemize}

Do stworzenia dwóch aplikacji zostanie użyte środowisko Android Studio.

\end{document}
