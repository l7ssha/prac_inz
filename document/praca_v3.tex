\documentclass[12pt,a4paper]{article}

\usepackage{minted}
\usepackage[T1]{fontenc}
\usepackage{url}

\renewcommand*\contentsname{Spis treści}

\title{Stacja pogodowa oparta na mikrokontrolerze ESP32 z interfejsem użytkownika oraz API}
\author{Szymon Uglis}

\begin{document}

% \begin{titlepage}
%     \maketitle
% \end{titlepage}

\tableofcontents{}
\pagebreak

\section{Wstęp}

// TODO:

\subsection{Cel projektu}

Celem projektu jest zbadanie przydatności modułu ESP32 oraz kompatybilnych komponentów do celów kolekcji danych pogodowych. 
Rezultatem będzie utworzenie urządzenia opartego na systemie wbudwanym pozwalającym na wielofunkcyjny pomiar parametrów pogodowych oraz udostępnianie ich za pomocą witryny www.

Urządzenie, będzie udostępniało również interfejs programistyczny API REST, które będzie umożliwiało integracje oraz dalszy rozwój projektu 
(np. integracja z systemem home assitant czy innymi urządzeniami Internetu rzeczy (IoT))

\subsection{Opis rozwiązania tworzonego w ramach projektu}

Narzędzie będzie składać sie z mikrokontrolera ESP-WROOM-32 oraz czujników pozwalająych kolekcje danych pogodowych (temperatury, ciśnienia, natężenia światła, wskaźnika UV)

\paragraph{Funkcje urządzenia}
\begin{itemize}
    \item Pomiar oraz kalkulacja danych pogodowych na podstawie danych wejściowych z czujników
    \item Udostępnienie i agregacja danych w postaci strony www
    \item Udostępnienie interfejsu programistycznego REST 
\end{itemize}

\section{Środowisko programowe}
Do implementacji systemu wbudowanego wykorzystane zostaną:
\begin{itemize}
    \item Język C do utworzenia oprogramowania mikrokontrolera
    \item Język HTML oraz CSS do utworzenia interfejsu www
    \item Środowisko programistyczne Arduino Studio w wersji 2.2.1
\end{itemize}

\section{Konfiguracja sprzętowa}

\begin{itemize}
    \item Mikrokontroler ESP-WROOM-32
    \item Czujnik natężenia światła - TSL25911
    \item Czujnik temperatury i wilgotności powietrza - DHT22
    \item Czujnik ciśnienia oraz temperatury - DPS310
    \item Czujnik światła ultrafioletowego - LTR390
\end{itemize}

\subsection{ESP-WROOM-32}
// TODO

\subsection{Czujnik natężenia światła - TSL25911}
// TODO

\subsection{Czujnik temperatury i wilgotności powietrza - DHT22}
// TODO

\subsection{Czujnik ciśnienia oraz temperatury - DPS310}
// TODO

\subsection{Czujnik światła ultrafioletowego - LTR390}

\section{Oprogramowanie}
// TODO

\subsection{Dostęp do danych przez przeglądarkę}
// TODO

\subsection{Dostęp do danych przez REST API}
// TODO

\section{Testy}
// TODO

\section{Wnioski}
// TODO

\section{Literatura}

\begin{enumerate}
    \item Dokumentacja czujnika TSL25911\\
    \url{https://learn.adafruit.com/adafruit-tsl2591}
    \item Dokumentacja czujnika DPS310\\
    \url{https://learn.adafruit.com/adafruit-dps310-precision-barometric-pressure-sensor}
    \item Dokumentacja czujnika DHT22\\
    \url{https://learn.adafruit.com/dht}
    \item Dokumentacja czujnika LTR390\\
    \url{https://learn.adafruit.com/adafruit-ltr390-uv-sensor}
    \item Dokumentacja mikrokontrolera ESP-WROOM-32\\
    \url{https://www.espressif.com/sites/default/files/documentation/esp32-wroom-32_datasheet_en.pdf}
\end{enumerate}

\section{Spisy rysunków, programów, ekranów, tabel, fotografii}
// TODO

\end{document}
