\documentclass[12pt,a4paper]{article}

\usepackage{minted}
\usepackage[T1]{fontenc}
\usepackage{url}
\usepackage{color,soul}

\renewcommand*\contentsname{Spis treści}

\title{Stacja pogodowa oparta na mikrokontrolerze ESP32 z interfejsem użytkownika oraz API}
\author{Szymon Uglis}

\begin{document}

% \begin{titlepage}
%     \maketitle
% \end{titlepage}

\tableofcontents{}
\pagebreak

\section{Wstęp}

\subsection{Wprowadzenie}

Dzięki szerokiemu dostępowi do internetu mamy łatwy dostęp do danych pogodowych aktualnych jak i historycznych z całego świata. Istnieje wiele
serwisów, programów telewizyjnych, które aktualne dane pogodowe, prognozy pogody prezentują nam w przystępny sposób.
Problemem z poleganiem na danych pogodowych z popularnych serwisów jest relatywnie niska dokładność aktualnych warunków pogodowych jak i prognoz pogody.
Bowiem serwisy te nie mogą mieć stacji pogowowych rozstawionych co na przykład kilometr, aby dokładność dla każdego potencjalnego zainteresowana była wysoka.
Dlatego polega się w dużym stopniu na ogólnych danych z kilku, bądź kilkunatstu stacji w danym regionie, aby wykonać ekstrapolację dla
aktualnych warunków pogodowych dla całego regionu. 
Serwisy udostępniające dane pogodowe dla lokalizacji są tylko przybliżeniem faktycznych stanu jaki znajduje sie w danym miejscu. 

\subsection{Cel projektu}

Celem projektu jest zbadanie przydatności modułu ESP32 oraz kompatybilnych komponentów do celów kolekcji danych pogodowych. 
Rezultatem będzie utworzenie urządzenia opartego na systemie wbudwanym pozwalającym na wielofunkcyjny pomiar parametrów pogodowych oraz udostępnianie ich za pomocą witryny www.

Urządzenie, będzie udostępniało również interfejs programistyczny API REST, które będzie umożliwiało integracje oraz dalszy rozwój projektu 
(np. integracja z systemem home assitant czy innymi urządzeniami Internetu rzeczy (IoT))

\subsection{Opis rozwiązania tworzonego w ramach projektu}

Narzędzie będzie składać sie z mikrokontrolera ESP-WROOM-32 oraz czujników pozwalająych kolekcje danych pogodowych (temperatury, ciśnienia, natężenia światła, wskaźnika UV)

\paragraph{Funkcje urządzenia}
\begin{itemize}
    \item Pomiar oraz kalkulacja danych pogodowych na podstawie danych wejściowych z czujników
    \item Udostępnienie i agregacja danych w postaci strony www
    \item Udostępnienie interfejsu programistycznego REST 
\end{itemize}

\section{Środowisko programowe}
Do implementacji systemu wbudowanego wykorzystane zostaną:
\begin{itemize}
    \item Język C do utworzenia oprogramowania mikrokontrolera
    \item Język HTML oraz CSS do utworzenia interfejsu www
    \item Środowisko programistyczne Arduino Studio w wersji 2.2.1
\end{itemize}

\section{Konfiguracja sprzętowa}

\begin{itemize}
    \item Mikrokontroler - ESP-WROOM-32
    \item Czujnik natężenia światła - TSL25911
    \item Czujnik temperatury i wilgotności powietrza - DHT22
    \item Czujnik ciśnienia oraz temperatury - DPS310
    \item Czujnik światła ultrafioletowego - LTR390
\end{itemize}

\subsection{ESP-WROOM-32}
ESP-WROOM-32 (albo ESP32-WROOM-32) to mikrokontroler ze zintegrowanym WIFi oraz bluetooth. Nadaje się do szerokiej gamy zastosowań:
od kontrolera czujników z niskim poborem energii do zaawansowanych zadań enkodowania sygnałów głosowych czy muzyki.

Moduły bazujące na rdzeniach ESP32 zyskały popularność ze wzklędu na zaintegrowaną obsługę WIFI, 
niskim kosztom oraz bogatej dokumentacji oraz wielu możliwych integracji.

Mikrokontroler znajdzie zastosowanie w prostych jak i bardziej skomplikowanych projektach, dzięki dwóm rdzenion, które mogą być kontrolowane osobno,
szerokich możliwościach podłączenia urządzeń perfyferyjnych (wsparcie dla: I2C, UART, SPI oraz inne).

ESP-WROOM-32 składa się z mikrokontrolera ESP32-D0WDQ6 posiadającego dwa rdzenie, które pozwają na prace od 80MHz do 240 MHz. Procesor również
posiada ko-procesor o niskim poborze mocy, który może zostać użyty zamiast głównych rdzeni w przypadku kiedy nie jest potrzebna duża moc obliczeniowa.

Integracja WIFI, Bluetooth i Bluetooth LE do mikrokontrolera pozwala na rozszerzenie możliwych zastosowań, gdzie nie jest możliwe podłączenie do sieci
w konwencjonalny spoób za pomocą kabla. Komunikacja bezprzewodowa również umożliwa całkowicie zdalne zastosowania z wykorzystaniem akumulatorów
oraz paneli słonecznych do zasilania mikrokontrolera.

// TODO: Uzasadnienie użycia ESP-32

\subsubsection{Specyfikacja ESP32}
\begin{tabular}{|l|l|}
    \hline
    Element & Specyfikacja \\
    \hline
    Procesor & 2 rdzenie (80-240MHz) \\
    \hline
    WIFI & 5802.11 b/g/n (802.11n: przepustowość 150 Mbps) \\
    \hline
    Bluetooth & Bluetooth v4.2 i Bluetooth LE \\
    \hline
    Pamięć SPI Flash & 4MB \\
    \hline
    Napięcie zasilania & 3.0 V ~ 3.6 V \\
    \hline
    Zużycie prądu & Średnie: 80 mA \\
    \hline
\end{tabular}

\subsection{Czujnik natężenia światła - TSL25911}
// TODO

\subsection{Czujnik temperatury i wilgotności powietrza - DHT22}
// TODO

\subsection{Czujnik ciśnienia oraz temperatury - DPS310}
// TODO

\subsection{Czujnik światła ultrafioletowego - LTR390}

\section{Oprogramowanie}
// TODO

\subsection{Dostęp do danych przez przeglądarkę}
// TODO

\subsection{Dostęp do danych przez REST API}
// TODO

\subsection{Dostęp do danych typu DEBUG}

Urządzenie udostępnia dane, które mogą słuzyć do debugowania urządzenia za pomocą interfejsu szeregowego. W przypadku wykorzystanego w projekcie
mikrokontrolera po uruchomieniu urządzenia dane zostaną wysyłane również na port szeregowy oprócz reszty procesów do przetwarzania danych.

Funkcja \texttt{send\_debug\_info\_to\_serial} wysyła dane na interfejs szeregowy w postaci łatwej do przeczytania:

\begin{minted}[frame=lines,]{c}
void send_debug_info_to_serial()
{
  Serial.print(F("DHT22 sensor data - Temp: '"));
  Serial.print(current_reading_data.temperature);
  Serial.print(F("' Humidity: '"));
  Serial.print(current_reading_data.humidity);
  Serial.print(F("' Heat index: '"));
  Serial.print(current_reading_data.heat_index);
  Serial.println(F("'"));

  Serial.print(F("TSL2591 sensor data - Calculated Lux: '"));
  Serial.print(current_reading_data.calculated_lux);
  Serial.print(F("' Full spectrum light: '"));
  Serial.print(current_reading_data.full_spectrum_light);
  Serial.print(F("' Infrared light: '"));
  Serial.print(current_reading_data.infrared_light);
  Serial.print(F("' Visible light: '"));
  Serial.print(current_reading_data.visible_light);
  Serial.println(F("'"));

  Serial.print(F("DPS310 sensor data - Pressure: '"));
  Serial.print(current_reading_data.pressure);
  Serial.println(F("hPa'"));

  Serial.print(F("LTR390 sensor data - UVS: '"));
  Serial.print(current_reading_data.uvs);
  Serial.println(F("'"));

  Serial.println(F(""));
}
\end{minted}

Jako, że powyższa funkcja operuje na dużej ilość tekstu w sposób cały, zostało zastosowana optymalizacja w postaci użycia funkcji 
\texttt{F} - funkcja ta sprawia, że dany łańcuch znaków nie jest kopiowany do pamięci PSRAM mikrokontrolera przed wysłaniem na interfejs szeregowy, 
a dane są beźpośrednio kopiowane z programu do strumienia, który wysyłane dane na interfejs. 

\section{Testy}
// TODO

\section{Wnioski}
// TODO

\section{Literatura}

\begin{enumerate}
    \item Dokumentacja czujnika TSL25911\\
    \url{https://learn.adafruit.com/adafruit-tsl2591}
    \item Dokumentacja czujnika DPS310\\
    \url{https://learn.adafruit.com/adafruit-dps310-precision-barometric-pressure-sensor}
    \item Dokumentacja czujnika DHT22\\
    \url{https://learn.adafruit.com/dht}
    \item Dokumentacja czujnika LTR390\\
    \url{https://learn.adafruit.com/adafruit-ltr390-uv-sensor}
    \item Dokumentacja mikrokontrolera ESP-WROOM-32\\
    \url{https://www.espressif.com/sites/default/files/documentation/esp32-wroom-32_datasheet_en.pdf}
\end{enumerate}

\section{Spisy rysunków, programów, ekranów, tabel, fotografii}
// TODO

\end{document}
