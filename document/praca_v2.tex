\documentclass[12pt,a4paper]{article}

\usepackage{minted}
\usepackage[T1]{fontenc}

\renewcommand*\contentsname{Spis treści}

\title{Narzędnie oparte na systemie wbudowanym umożliwiające wielofunkcyjny pomiar światła w ciemni fotograficznej}
\author{Szymon Uglis}

\begin{document}

\begin{titlepage}
    \maketitle
\end{titlepage}

\tableofcontents{}
\pagebreak

\section{Wstęp}

// TODO:

\section{Cel projektu}

Celem projektu jest zbadanie przydatności modułu ESP32 oraz kompatybilnych komponentów do celów wytwarzanie powiększeń w ciemni fotograficznej.
Rezultatem będzie utworzenie urządzenia opartego na systemie wbudwanym pozwalającym na wielofunkcyjny pomiar światła 
oraz wykonywanie obliczeń w kilku różnych trybach na danych otrzymanych z czujników. Urządzenie będzie udostępniało interfejs w postaci witryny www co umożliwi
prosty dostęp do danych obliczanych oraz udstępnianych przez urządzenie.

Urządzenie, będzie udostępniało interfejs programistyczny API REST, które będzie umożliwiało integracje oraz dalszy rozwój projektu 
(np. integracja z systemem home assitant czy innymi urządzeniami Internetu rzeczy [IoT])

\section{Opis rozwiązania tworzonego w ramach projektu}

Narzędzie będzie składać sie z mikrokontrolera ESP-WROOM-32, czujnika światła TSL25911 oraz urządzenie będzie udostępniać stronę internetową do obsługi oraz zmiany parametrów pracy.
\begin{enumerate}
    \item Funkcje urządzenia
    \begin{itemize}
        \item Pomiar oraz kalkulacja rozpiętości tonalej planowanego powiększenia
        \item Pomiar rozposzonego światła pozwalającego na ustalenie prawidłowego czasu naświetlenia powiększenia
        \item Kalkulacja czasu naświetlenia powiększenia w przypadku zmiany formatu
    \end{itemize}
    \item Strona internetowa do sterowania urządzeniem
    \begin{itemize}
        \item Wyświetlenie rezultatów obliczeń
        \item Wyświetlenie aktualnych ustawień urządzenia
        \item Możliwość zmiany ustawień oraz paremetrów do obliczeń
    \end{itemize}
\end{enumerate}

\subsection{Wykorzystane technologie}
Do implemntacji systemu wbudowanego wykorzystane zostaną:
\begin{itemize}
    \item Język C do utworzenia oprogramowania mikrokontrolera
    \item Środowisko programistyczne Arduino Studio
    \item Język HTML oraz CSS do utworzenia interfejsu www
\end{itemize}

\subsection{Zastosowany mikrokontroler oraz czujniki}

\begin{itemize}
    \item Mikrokontroler ESP-WROOM-32
    \item Czujnik natężenia światła TSL25911
\end{itemize}

\subsection{Opis 7W}

// TODO: 

\subsection{Wymagania niefunkcjonalne}

// TODO: 

\subsection{Przypadki użycia}

\begin{itemize}
    \item Znany czas naświetlenia powiększenia w formacie 4x5, chęć utworzenia fotografi 8x10
    \begin{itemize}
        \item Za pomocą interfejsu www zmiana trybu pracy na tryb kalkulacji czasu naświetlania w przypadku zmiany formatu oraz wprowadzenie znanego czasu dla aktualnego formatu
        \item Pomiar rozproszonego światła dla formatu 4x5 (referencyjny pomiar dla znanego czasu)
        \item Pomiar rozproszonego światła dla formatu 8x10 (referencyjny pomiar potrzebny do kalkulacji faktycznego czasu dla innego formatu)
        \item Odczytanie czasu obliczonego przez urządzenie w interfejscie www
    \end{itemize}
    \item Pomiar kontrastu papieru
    \begin{itemize}
        \item Za pomocą interfejsu www zmiana trybu pracy na tryb pomiaru kontrastu papieru oraz wprowadzenie rodzaju aktualnie używanego papieru
        \item Pomiar światła dla najjaśniejszego elementu fotografii rzucanego przez powiększalnik
        \item Pomiar światła dla najciemniejszego elementu fotografii rzucanego przez powiększalnik
        \item Odczytanie wymaganego kontrastu papieru w interfejscie www
    \end{itemize}
\end{itemize}


\end{document}
